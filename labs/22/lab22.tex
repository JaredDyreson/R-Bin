Math 338 \textbf{Lab Assignment \#22} Fall 2019

Recall from lecture that confidence intervals come out of Neyman-Pearson
ideas. However, instead of making a choice between {two} possible
parameter values, we estimate an entire {interval} of values within
which we suspect the true parameter value lies. Thus, we can use our
confidence interval estimates to do two-sided hypothesis tests in the
NHST framework.

First, let's re-create the setosa.petal.length dataset:

\textbf{\textgreater{} library(dplyr)}

\textbf{\textgreater{} setosa.petal.length \textless- iris
\%\textgreater\% filter(Species == "setosa") \%\textgreater\%
select(Petal.Length)}

Recall that we used this dataset to test the null hypothesis that the
population mean petal length was 1.3 cm. Now we will test this null
hypothesis using a confidence interval instead.

\textbf{\textgreater{} t.test(setosa.petal.length\$Petal.Length,
conf.level = 0.95) \# technically the confidence level is 0.95 by
default}

\textbf{Question \#1} Paste the output from R. Write a sentence
interpreting the 95\% confidence interval for the population mean petal
length of \emph{setosa} irises.

\textbf{Question \#2} Based on our confidence interval, does the
population mean petal length of \emph{setosa} irises appear to be
greater than 1.3 cm or less than 1.3 cm? Or can you not tell? Explain
your answer.

\textbf{Question \#3} Based on our confidence interval, can we reject
H\textsubscript{0}: μ = 1.3 in favor of the two-sided NHST alternative
H\textsubscript{a}: μ ≠ 1.3 at the 5\% significance level? Why or why
not?

Now, let's look at the difference of population mean sepal lengths
between \emph{setosa} and \emph{versicolor}. We'll need the original
\emph{iris} dataset for this one. Since our \emph{setosa} and
\emph{versicolor} samples are unrelated, this is a two-sample \emph{t}
confidence interval.

\textbf{\textgreater{} iris\_sv \textless- iris \%\textgreater\%
filter(Species \%in\% c("setosa","versicolor"))}

\textbf{\textgreater{} t.test(Sepal.Length \textasciitilde{} Species,
data = iris\_sv, conf.level = 0.95)}

\textbf{Question \#4} Paste the output table from R. Write a sentence
interpreting the 95\% confidence interval for the difference of
population mean sepal lengths between setosa and versicolor.

\textbf{Question \#5} Based on our confidence interval, which species
has longer petals? Or can you not tell? Explain your answer.

\textbf{Question \#6} Based on our confidence interval, can we reject
H\textsubscript{0}: \(\mu_{1} - \mu_{2} = 0\) in favor of the two-sided
NHST alternative H\textsubscript{a}: \(\mu_{1} - \mu_{2}\  \neq 0\) at
the 5\% significance level? Why or why not?

Finally, we will do a \emph{t} confidence interval for the population
mean of paired differences (matched pairs confidence interval). We used
the straight\_jeans2 dataset for this one and investigated the mean of
the paired differences in maximum back pocket width
(\emph{maxWidthBackMens} and \emph{maxWidthBackWomens}).

There's actually a way to do a paired test without getting the
difference first; we just almost always look at a histogram of the
differences first. Just to get the practice, we'll use a 99\% CI.

\textbf{\textgreater{} t.test(x = straight\_jeans2\$maxWidthBackMens, y
= straight\_jeans2\$maxWidthBackWomens, paired = TRUE, conf.level =
0.99)}

or (if you like to do the subtraction first and then just run the
one-sample test) you could copy the code from Lab 19 and include the
confidence level argument.

\textbf{\textgreater{} jeans\_diff \textless- straight\_jeans2
\%\textgreater\% mutate(maxWidthBackDiff = maxWidthBackMens -
maxWidthBackWomens)}

\textbf{\textgreater{} t.test(jeans\_diff\$maxWidthBackDiff, conf.level
= 0.99)}

\textbf{Question \#7} Paste the output from R. Write a sentence
interpreting the 99\% confidence interval for the population mean of the
paired differences in maximum back pocket width.

\textbf{Question \#8} Based on our confidence interval, whose jeans have
wider back pockets -- men or women? Or can you not tell? Explain your
answer.

\textbf{Question \#9} Based on our confidence interval, can we reject
H\textsubscript{0}: μ\textsubscript{d} = 0 in favor of the two-sided
NHST alternative H\textsubscript{a}: μ\textsubscript{d} ≠ 0 at the 1\%
significance level? Why or why not?

\textbf{Question \#10} Based on your answers in this lab, what
information can you get from a {confidence interval} that you cannot get
from a hypothesis test?
